\documentclass{proc}
\title{\sf EDAN55 Kurssammanfattning}
\usepackage{mdframed}

\begin{document}
\maketitle

\tableofcontents

\section{Set Cover (with weights)}

Find the collection such that the union of its sets are equal to all elements in U. In weighted version, every set $S_{i}$ has an associated weight $w_{i} \ge 0$.

\begin{mdframed}
    \textbf{Example:}
    
    U = \{$S_{1}, S_{2}, S_{3}, S_{4}$\} = \{\{1,2\}, \{2\}, \{2,3\}, \{3,4,5\}\}
    
    $S_{1} \cup S_{4} = \{1,2,3,4,5\}$ = All elements of U = Set Cover of U.
\end{mdframed}

\textbf{Goal:} Find set cover C such that $\sum_{S_{i} \in C} w_{i}$ is minimised.

Maintain a set R of all remaining uncovered elements.
\begin{mdframed}
    $\frac{w_{i}}{|S_{i} \cap R|} = $ cost for covering remaining elements in $S_{i}$.
\end{mdframed}

\begin{mdframed}
    \textbf{Algorithm:} (Greedy-Set-Cover)
    
    R = U
    
    While $R \neq \emptyset$
    
        \hspace{2ex} Select $S_{i}$ which minimises $\frac{w_{i}}{|S_{i} \cap R|}$.
        
        \hspace{2ex} Delete all $s \in S_{i}$ from R.
        
    End
    
    Return selected sets
\end{mdframed}

\begin{mdframed}
    \textbf{Example:} (bad instance)
    
    $S_{1} = \{1,2,3,4\}, w_{1} = 1 + \epsilon$ ($\epsilon$ small number $> 0$)
    
    $S_{2} = \{5,6,7,8\}, w_{2} = 1 + \epsilon$
    
    $S_{3} = \{3,4,7,8\}, w_{3} = 1$
    
    $S_{4} = \{2, 6\}, w_{4} = 1$
    
    $S_{5} = \{1\}, w_{5} = 1$
    
    $S_{6} = \{5\}, w_{6} = 1$
    
    \vspace{2ex}
    
    \textit{Optimal solution: $2 + 2\epsilon$ $\{S_{1}, S_{2}\}$.}
    
    \vspace{2ex}
    
    1. Picks $S_{3}$ instead of $S_{1}$ or $S_{2}$ since $\frac{1}{4} < \frac{1 + \epsilon}{4}$.
    
    2. Picks $S_{4}$ instead of $S_{1}$ or $S_{2}$ since $\frac{1}{2} < \frac{1 + \epsilon}{2}$
    
    2. Picks $S_{5}$ or $S_{6}$ instead of $S_{1}$ or $S_{2}$ since $\frac{1}{1} < \frac{1 + \epsilon}{1}$
    
    Finds solution: $1 + 1 + \frac{1}{2} + \frac{1}{4} = 2.75 > 2 + 2\epsilon$.
    
    This example can be expanded in the same way to construct an arbitrarily large instance that will perform just as bad.
\end{mdframed}

\begin{mdframed}
    \textbf{Record cost:}

    $c_{s} := \frac{w_{i}}{S_{i} \cap R}$ for every $s \in S_{i} \cap R$
    
    \textit{Does not change the algorithm, used only for analyse.}
\end{mdframed}

\begin{mdframed}
    \textbf{Example:}
    
    $S_{1} = \{1, 3\}, w_{1} = 1$
    
    $S_{2} = \{2,3,4\}, w_{1} = 1$
    
    1. Pick $S_{2}$ of cost $\frac{1}{3} \Rightarrow c_{2} = c_{3} = c_{4} = \frac{1}{3}$
    
    2. Pick $S_{1}$ of cost $\frac{1}{1} \Rightarrow c_{1} = 1, (c_{3} = \frac{1}{3}$, unchanged)
\end{mdframed}

The costs completely account for the total weight of the set cover.

\begin{mdframed}
    \textbf{(11.9)} If C is the set cover obtained by the Greedy-Set-Cover, then $\sum_{S_i \in C} W_i = \sum_{s \in U} c_s$.
\end{mdframed}

\begin{mdframed}
    \textbf{Example:} (continuation of previous)

    $\sum_{S_i \in C} W_i = w_{1} + w_{2} = 1 + 1 = 2$
    
    $\sum_{s \in U} c_s = c_1 + c_2 + c_3 + c_4 = \frac{1}{3} + \frac{1}{3} + \frac{1}{3} + 1 = 2$
    
    $\Rightarrow \sum_{S_i \in C} W_i = \sum_{s \in U} c_s$
\end{mdframed}

\textbf{Q:} How much cost can any single set $S_k$ account for, including sets not picked by the algorithm? Find upper bound for the ratio $\frac{\sum_{s \in S_k} c_s}{w_k}$.

The optimum solution must cover the full cost $\sum_{s \in U} c_s$ via the sets it selects so this bound will establish that it needs to use at least a certain amount of weight ( = lower bound, what we want).

\begin{mdframed}
    \textbf{Harmonic function:}
    
    $H(n) = \sum_{i=1}^{n} \frac{1}{i}$
    
    Sum approximates the area under the curve $y = \frac{1}{x}$.
    
    Naturally bounded above by $1 + \int_{1}^{n} \frac{1}{x} dx = 1 + ln(n)$ and
    
    below by $\int_{1}^{n+1} \frac{1}{x} dx = ln(n + 1)$.
    
    Thus $H(n) = \Theta(ln(n))$
\end{mdframed}

\begin{mdframed}
    \textbf{(11.10)} For every set $S_k$, the sum $\sum_{s \in S_k} c_s$ is at most $H(|S_k|)*w_k$.
    
    \textbf{Proof.}
    Simplify notation by assume that the elements of $S_k$ are the first $d = |S_k|$ elements of the set U ($S_k = \{s_1,\ldots, s_d\}$).
    
    \begin{mdframed}
        \textbf{Example:}
    
        $U = \{a,b,c,d,e,f,g\}$
    
        $S_1 = \{a, b\}, d = |S_1| = 2$
    
        $S_2 = \{a,b,c,d,e\}, d = |S_2| = 5$
    \end{mdframed}
    
    Assume that the elements are labeled in the order in which they are assigned a cost $c_{s_j}$ by the greedy algorithm (ties broken arbitrarily). No loss of generality, only involves renaming elements in U.
    
    Consider the iteration when $s_j$ is covered by the greedy algorithm for some $j \le d$. When the iteration begins, $s_j, s_{j+1}, \ldots, s_d \in R$, according to our naming convention (elements have not been selected). This implies that $|S_k \cap R| \ge d - j + 1$ (there are $d - j + 1$ not already selected elements in $S_k$). The average cost of the set $S_k$ is at most $\frac{w_k}{|S_k \cap R|} \leq \frac{w_k}{d-j+1}$.
    
    It is not necessarily an equality because in the same iteration as $s_j$ is covered by the greedy algorithm, some other elements $s_{j'}$ for $j' < j$ may be covered as well.
    
    In this iteration, the greedy algorithm selects a set $S_i$ of a minimum average cost so that the set $S_i$ has an average cost at most that of $S_k$. The average cost of $S_i$ gets assigned to $s_j$, so
    
    $c_{s_j} = \frac{w_i}{|S_i \cap R|} \le \frac{w_k}{|S_k \cap R|} \le \frac{w_k}{d-j+1}$.
    
    Add up all inequalities for all elements $s \in S_k$:
    
    $\sum_{s \in S_k} c_s = \sum_{j=1}^{d} c_{s_j} \le \sum_{j=1}^{d} \frac{w_k}{d-j+1} = \frac{w_k}{d} + \frac{w_k}{d-1} + \ldots + \frac{w_k}{1} = H(d) * w_k$
\end{mdframed}

Let $d* = max_i |S_i|$ denote the maximum size of any set.

\begin{mdframed}
    \textbf{(11.11)} The set cover C selected by the Greedy-Set-Cover has weight at most H(d*) times the optimal weight w*.
    
    \textbf{Proof.} Let C* denote the optimum weight set cover, so that $w* = \sum_{S_i \in C*} w_i$. For each of the sets in C*, (11.10) implies
    
    $w_i \ge \frac{1}{H(d*)} \sum_{s \in S_i} c_s$
    
    (The cost has to be paid by the weight). Because these sets form a set cover, we have
    
    $\sum_{S_i \in C*} \sum_{s \in S_i} c_s \ge \sum_{s \in U} c_s$
    
    (The same element can be present several times in C* but not in U, there for no equality). Combining these with (11.9) gives the desired bound:
    
    $w* = \sum_{S_i \in C*} w_i \ge \sum_{S_i \in C*} \frac{1}{H(d*)} \sum_{s \in S_i} c_s \ge \frac{1}{H(d*)} \sum_{s \in U} c_s = \frac{1}{H(d*)} \sum_{S_i \in C} w_i$
\end{mdframed}

The greedy algorithm finds a solution within a factor $O(log(d*)$ of the optimal. Since the maximum set size $d*$ can be a constant fraction of the total number of elements n, this is a worst-case upper bound of $O(log(n))$. By expressing the bounds in terms of $d*$ shows us that we're doing much better if the largest set is small.

It has been shown that no polynomial-time approximation algorithm can achieve an approximation bound much better than $H(n)$, unless P = NP. 

\section{The Pricing Method: Vertex Cover}

Vertex cover in a graph $G = (V,E)$ is a set $S \subseteq V$ so that each edge has at least one end in S. In this version of the problem, each vertex $i \in V$ has a weight $w_i \ge 0$, with the weight of a set $S$ of vertices denoted $w(S) = \sum_{i \in S} w_i$. 

\begin{mdframed}
    \textbf{Goal:} find a vertex cover S for which w(S) is minimised.
\end{mdframed}

(When all weights are equal to 1, deciding if there is a vertex cover of weight at most $k$ is the standard decision version of Vertex Cover).

\begin{mdframed}
    Vertex Cover $\le_{P}$ Set Cover
\end{mdframed}

If we had a polynomial-time algorithm that solves the Set Cover Proble, then we could use this algorithm to solve the Vertex Cover Problem in polynomial time.

\begin{mdframed}
    \textbf{(11.12)} The Set Cover approximation algorithm can be used to give an $H(d)$-approximation algorithm for the weight Vertex Cover Problem, where d is the maximum degree of the graph. (The degree of the graph is the maximum number edges attached to a vertex).
\end{mdframed}

\end{document}